% !TEX program = xelatex

\documentclass{beamer}
\usepackage[utf8]{inputenc}
\usepackage{xeCJK}
\usepackage{utopia} %font utopia imported
%\usepackage[UTF8,noindent]{ctexcap}
\usetheme{Madrid}
\usecolortheme{default}
 % 衬线字体:Linux Libertine
    % BoldFont 可以选择 Bold 字重或者 Semibold 字重
    % BoldItalicFont 也有对应 BoldFont 的字重选择
    % 这里使用 Semibold 字重
  %   \setmainfont{LinLibertine_R.otf}[
  %     BoldFont = LinLibertine_RZ.otf,
  %     ItalicFont = LinLibertine_RI.otf,
  %     BoldItalicFont = LinLibertine_RZI.otf, ]
  % % 无衬线字体:Linux Biolinum
  % \setsansfont{LinBiolinum_R.otf}[
  %     BoldFont = LinBiolinum_RB.otf,
  %     ItalicFont = LinBiolinum_RI.otf,
  %     BoldItalicFont = LinBiolinum_RBO.otf]
  % 等宽/打印机字体:Linux Libertine Mono
  % \setmonofont{LinLibertine_M.otf}[
  %     BoldFont = LinLibertine_MB.otf,
  %     ItalicFont = LinLibertine_MO.otf,
  %     BoldItalicFont  = LinLibertine_MBO.otf]
  % \setCJKmainfont[ItalicFont={AR PL UKai CN},
  % BoldFont={WenQuanYi Micro Hei}]{IPAMinCho,IPA明朝}
  \setCJKsansfont{WenQuanYi Micro Hei}
  \setCJKmonofont{WenQuanYi Micro Hei Mono}
%------------------------------------------------------------
%This block of code defines the information to appear in the
%Title page
\title[光电中期报告] %optional
{光电子技术实验}

\subtitle{固体激光器的静态特性及调Q技术}

\author[芦, 王] % (optional)
{芦迪 \and 王莘景}

\institute[THU, EE] % (optional)
{

  Department of Electronic Engineering\\
  Tsinghua University
}

\date[2017.11.7] % (optional)
{\today}

\logo{\includegraphics[height=1.5cm]{thuee-logo.png}}

%End of title page configuration block
%------------------------------------------------------------



%------------------------------------------------------------
%The next block of commands puts the table of contents at the 
%beginning of each section and highlights the current section:

\AtBeginSection[]
{
  \begin{frame}
    \frametitle{目录}
    \tableofcontents[currentsection]
  \end{frame}
}
%------------------------------------------------------------


\begin{document}

%The next statement creates the title page.
\frame{\titlepage}


%---------------------------------------------------------
%This block of code is for the table of contents after
%the title page
\begin{frame}
\frametitle{目录}
\tableofcontents
\end{frame}
%---------------------------------------------------------


\section{实验任务}

%---------------------------------------------------------
%Changing visivility of the text
\begin{frame}
\frametitle{实验任务}

\begin{enumerate}
    \item 装调固体激光器使之产生激光,反复调整降低阈值
    \item 测量固体激光器输出-输入能量关系曲线
    \item 观察激光器的静态输出波形,
    \item Text visible on slide 4
\end{enumerate}
\end{frame}

%---------------------------------------------------------


%---------------------------------------------------------
%Example of the \pause command
\begin{frame}
In this slide \pause

the text will be partially visible \pause

And finally everything will be there
\end{frame}
%---------------------------------------------------------

\section{实验原理}

%---------------------------------------------------------
%Highlighting text
\begin{frame}
\frametitle{Sample frame title}

In this slide, some important text will be
\alert{highlighted} beause it's important.
Please, don't abuse it.

\begin{block}{Remark}
Sample text
\end{block}

\begin{alertblock}{Important theorem}
Sample text in red box
\end{alertblock}

\begin{examples}
Sample text in green box. "Examples" is fixed as block title.
\end{examples}
\end{frame}
%---------------------------------------------------------


%---------------------------------------------------------
%Two columns
\begin{frame}
\frametitle{Two-column slide}

\begin{columns}

\column{0.5\textwidth}
This is a text in first column.
$$E=mc^2$$
\begin{itemize}
\item First item
\item Second item
\end{itemize}

\column{0.5\textwidth}
This text will be in the second column
and on a second tought this is a nice looking
layout in some cases.
\end{columns}
\end{frame}
%---------------------------------------------------------
\section{实验系统}
\section{方法步骤}
\section{实验结果}
\section{结果分析及结论}


\end{document}